% This section should provide details of how the prototype has been implemented which may involve presentation of suitable code snippets.
\section{Prototype Implementation}
\label{sec:implementation}
This section will describe how the different software technologies were implemented and how they were used in this project, by briefly mentioning what they do and thoughts behind implementing them. Code snippets are also showed to explain and visualize how the code looks in a real-time project.

\subsection{JPA}\label{sub:jpa}
Prototype implementation had few major steps. First and most crucial one, was to create JPA entities and let them communicate with each other, from given relations between them.

Two entities were created, \textbf{POLL} and \textbf{USER} with appropriate relations stored in a H2 relational database. Relations between those entities are One-To-Many, since one user can create many polls, but only one user can be the creator/owner of a specific poll. Furthermore, there is a possibility to vote on a specific poll, thus there is a need for storing data about that as well. This is handled with a "Many-To-Many" relation between Poll and User entities, creating a new table with two attributes \textbf{User.uID} and \textbf{Poll.pollID} storing "who voted on what".

As shown in the example below, JPA makes it very easy to create new entities. Just by adding annotation \textit{@Entity}, the Java class is then declared to behave as an entity when program runs. In the same way \textit{@Id, @ManyToOne, @ManyToMany, ...} annotations can be used to configure the desired relations and identities for different entity attributes.\vspace{\baselineskip}
\lstinputlisting[language=java]{code/pollentity.java}\vspace{\baselineskip}

JPA doesn't give basic; create, read, update and delete (CRUD) operations, when new entities are created. Therefore, it was necessary to create own Data Access Object (DAO) for every entity created. This is very cumbersome and time consuming. Benefits of using Spring Framework, is that CRUD operation can very simply be added for entities by extending a Spring Framework Class (CrudRepository or PagingAndSortingRepository which extends CrudRepository). PagingAndSortingRepository was chosen in this project, since it provides the same functionalities as CrudRepository Class, and extra functions which make it effortless to "page and sort" the obtained results.  

In the code snippet below, the implementation of User repository extending the Spring Framework CRUD operations class can be seen. With an extra functionality, \textit{findByFname}, which is a read operation to search for a user by his/her first name.\vspace{\baselineskip}
\lstinputlisting[language=java]{code/userrepo.java}

\subsection{REST}\label{sub:rest}
With the entities set up and ready to be used, a user needs some way to interact with this data. To accomplish this, a REST API needs to be implemented. To implement a business logic side, Spring Framework was again used. With the benefits of using Java annotations \textit{@Controller}, the Java class is then declared to behave as a RestController when program runs. Later by creating new methods within this Class and annotating them with desired mapping annotation, all CRUD operations can be accessed by HTTP requests.\vspace{\baselineskip}
\lstinputlisting[language=java]{code/pollrest.java}\vspace{\baselineskip}

As shown in the code snippet above, it is very simple to use Spring frameworks annotation to also make a REST controller. The \textit{@Controller} annotation, makes this class a controller when the program runs. In order to use different repositories in this controller class, there needs to be a way to access them. This is accomplished with the annotation \textit{@Autowired}, that wires the repository to the controller when the program launches and data from it can be accessed.

In a controller class different request mappings can be used, such as \textit{@GetMapping} and \textit{@PostMapping}, that were used in this back end code. In the different request mappings, there is a need to define the path at which to access the different methods. In the code snippet there are two examples of such mappings, one POST mapping, that at the path \textit{/pollAdd}, needs parameters defined with \textit{@RequestParam} to execute the \textit{pollAdd()} method. And a second, GET mapping to search polls at the path \textit{/polls} with defined parameters.

While implementing REST API to this prototype; post, get, put, delete mappings were used in order to make user interact with prototype's business logic. Since this requires the user to send HTTP-request with different mappings for different operations. Later, this was modified and downgraded to only GET and POST mappings, considering those are the only two HTTP requests a HTML form supports, and can send as a HTTP request.

\subsubsection{IoT support}\label{subsub:IoTsupport}
Final project has support for a voting IoT device and display IoT device, to connect to a poll with HTTP request. To make this a possibility, more CRUD operation mappings were added to the program. When suitable GET and POST mappings were implemented such that an IoT device could connect to a poll, a program was created in Java to simulate an IoT device.

A small and easy Java program was created with simple HTTP requests for the IoT to connect to the prototype. In order to create the program, just one Java class was needed and few helping methods to check for any potential errors when "user" input is used to navigate in the program and vote on a poll or look-up its results.

\subsection{Security}\label{sub:security}
Since a user must register and login to use most parts of the web page, there needed to be some security implemented. For this technology stack, Spring frameworks Security dependency was needed, that allowed to simply define pages accessible and not accessible without logging in. This can be seen in the code snippet below.

For a “simpler” and more secure login process, Google's OAuth 2 was used. By doing this, there is no need to store sensitive user information, and user doesn't need to create a new password for him/her to remember. As the best security is the security one doesn't need to self implement, which can cause unintended errors or follow with security holes.\vspace{\baselineskip}
\lstinputlisting[language=java]{code/security.java}

{\footnotesize\textbf{Disclaimer!}\textit{ All documentation and tutorials looked at about Spring Framework for this project, were taken from \cite{spring_Framework}}.}

\subsection{Firebase}\label{sub:firebaseimplement}
The project required storing of results in a noSQL database. For this task Firebase Cloud Firestore was chosen. The way this was implemented, was by creating a service component by adding the @Service annotation from Spring Boot. This class handles initializing of the Firebase database, so Spring \textit{@Postconstruct} is also added, to make sure the method is initialized properly, even before the class is put into service.

For population of the database with results from the poll, \textit{@PostMapping} and \textit{@PutMapping} were used from the controller. When a poll is created, it momentarily gets stored in the database. Same goes for the updates, when for example someone is voting on a poll, it gets updated in the database as the vote is cast. In Listing~\ref{code:firebase} one can see a code snippet for how the poll gets stored in the database, with collection name Polls. It then stores the poll created with document name set as the Polls ID.
\lstinputlisting[label={code:firebase},caption={Adding poll to Firebase}, language={java}]{code/firebase.java}
\subsection{Messaging System}\label{sub:messaging}
\subsubsection{RabbitMQ}
In the web application, a user can subscribe to a poll and receive its results when the poll closes. This calls for a messaging system implantation. RabbitMQ is used to make this possible, with a receiver and a sender class, again using Java annotation \textit{@RabbitListener} to declare a method to be a receiver. A topic-based messaging queue seemed most relevant in this case to use, as it made it easy to use poll status (present, past, future) as the topic key.

But before a user can receive messages from a subscription, the user must subscribe to a specific poll, that is then saved in a subscription list that the user has saved in its entity. From these lists, rabbit listener finds correct receivers and sends the requested message to them.
\subsubsection{Dweet.io}
As for the implementation of dweet.io, it was desired that the events of starting and closing a poll should be “dweeted”. This simply means that whenever a poll is created, information about the poll name and description is published to dweet.io. A poll that is closed will additionally send information about the result of the poll. When implementing code that will send out a dweet, it is important that a unique name/ID is chosen. If a machine or user wants to get information about this dweet, it will use this unique name to get it.

The code snippet in Listing~\ref{code:dweet} illustrates how the application would extract the poll name and description, then post it to dweet.io, in the events of creating a new poll. After setting a unique name, it takes the name and description values and puts them into a JSON element. It is lastly published to dweet.io by using the publish function provided by the dweet.io library for Java, with the unique name and JSON element inputted as parameters. If one wished to get this dweet, one could simply type a request URL with the unique name in a browser. In this case, it would be: \textit{https://dweet.io:443/get/latest/dweet/for/unique\_poll\_dweet.}
\lstinputlisting[label={code:dweet},caption={Sending a dweet of a newly created poll with name and description}, language={java}]{code/dweet.java}

\subsection{Front End}\label{sub:frontend}
User interacts with the program via web browser and gets information from database by communicating with business logic via Spring MVC, using Thymeleaf. Thymeleaf is an integrated method to use in HTML code in order to access data from models, generated based on the request sent by user. HTML code was used to create different use case screens and according to what requests can be made, the appropriate thymeleaf code was inserted into HTML code. For the pages to look "better" and be user friendly with what can be done on a web page, CSS was used to decorate the pages and how they look/show up for the client.